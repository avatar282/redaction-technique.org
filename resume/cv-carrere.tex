%%%%%%%%%%%%%%%%%%%%%%%%%%%%%%%%%%%%%%%%%
% "ModernCV" CV and Cover Letter
% LaTeX Template
% Version 1.11 (19/6/14)
%
% This template has been downloaded from:
% http://www.LaTeXTemplates.com
%
% Original author:
% Xavier Danaux (xdanaux@gmail.com)
%
% License:
% CC BY-NC-SA 3.0 (http://creativecommons.org/licenses/by-nc-sa/3.0/)
%
% Important note:
% This template requires the moderncv.cls and .sty files to be in the same
% directory as this .tex file. These files provide the resume style and themes
% used for structuring the document.
%
%%%%%%%%%%%%%%%%%%%%%%%%%%%%%%%%%%%%%%%%%

%-------------------------------------------------------------------------------
%	PACKAGES AND OTHER DOCUMENT CONFIGURATIONS
%-------------------------------------------------------------------------------

\documentclass[11pt,a4paper,roman]{moderncv}
% Font sizes: 10, 11, or 12; paper sizes: a4paper, letterpaper, a5paper,
% legalpaper, executivepaper or landscape; font families: sans or roman

\moderncvstyle{casual}
% CV theme - options include: 'casual' (default), 'classic', 'oldstyle' and
% 'banking'

\moderncvcolor{blue}
% CV color - options include: 'blue' (default), 'orange', 'green', 'red',
% 'purple', 'grey' and 'black'

\usepackage[width=45em,height=60em]{geometry}

% Reduce document margins \setlength{\hintscolumnwidth}{3cm} Uncomment to change
% the width of the dates column \setlength{\makecvtitlenamewidth}{10cm} For the
% 'classic' style, uncomment to adjust the width of the space allocated to your
% name

\usepackage[francais]{babel}
\usepackage[utf8]{inputenc}
\DeclareUnicodeCharacter{00A0}{\nobreakspace}

%-------------------------------------------------------------------------------
%	NAME AND CONTACT INFORMATION SECTION
%-------------------------------------------------------------------------------

\firstname{Olivier} % Your first name
\familyname{Carrère} % Your last name

% All information in this block is optional, comment out any lines you don't
% need
\title{Rédacteur technique anglais français}
\address{}{93100 Montreuil}
\mobile{}
\email{carrereo@gmail.com}

\homepage{www.redaction-technique.org}
         {www.redaction-technique.org}

\extrainfo{github.com/olivier-carrere}
% The first argument is the url for the clickable link, the second argument is
% the url displayed in the template - this allows special characters to be
% displayed such as the tilde in this example

         \quote{Gérer la documentation comme le code source.}

%-------------------------------------------------------------------------------

\begin{document}

\makecvtitle % Print the CV title

%-------------------------------------------------------------------------------
%	WORK EXPERIENCE SECTION
%-------------------------------------------------------------------------------

\section{Expérience}

\subsection{Responsable de la communication technique}

\cventry{Depuis octobre 2012}
        {Arealti}
        {SSII}
        {}
        {}
        {Industrialisation de la documentation de l'éditeur de solutions réseau
          6WIND.
          \begin{itemize}
          \item Développement d'une plateforme de documentation multi-formats
            basée sur le langage de balisage léger (de type \textbf{Wiki} ou
            \textbf{Markdown}) \textbf{ReStructuredText.}
          \item Augmentation de la productivité par la modularisation et la
            factorisation du contenu et la mise en place de mécanismes de texte
            conditionnel
          \item Suivi des modifications \textit{via} le système de gestion de
            bugs \textbf{Bugzilla}
          \item Gestion fine des versions \textit{via} le logiciel de gestion de
            versions \textbf{Git}
          \item \textit{Backport} des modifications dans les branches de
            maintenance
          \item Création de feuilles de style \textbf{CSS} (HTML) et
            \textbf{\LaTeX} (PDF)
          \item Modification du contenu en mode interactif ou \textit{batch}
        \end{itemize}
    }

%------------------------------------------------

\cventry
    {2011 - 2012}
    {Rédacteur technique freelance}
    {\textsc{}}
    {}
    {}
    {Missions de conseil et de formation \textbf{DITA XML.}}

%------------------------------------------------

\cventry
    {2009 – 2011}
    {Edenwall}
    {Pare-feu identifiant}
    {}
    {}
    {Mise en place des process de création et mise à jour de la documentation
      basés sur l'architecture modulaire DITA XML.
      \begin{itemize}
      \item Mise en place d'une base documentaire modulaire DITA XML
      \item Gestion des versions sous \textbf{Subversion}
      \item Gestion des tickets sous \textbf{Trac}
      \item Création de feuilles de style \textbf{XSLT}
      \item Manuels d'installation, guides de l'utilisateur et livres blancs en
        anglais
    \end{itemize}}

%------------------------------------------------

\cventry
    {2006 – 2009}
    {OpenTrust}
    {Logiciels de sécurité}
    {}
    {}
    {Documentation de logiciels de PKI, de signature électronique et
      de gestion de carte à puces.
      \begin{itemize}
      \item Mise en place d'une chaîne de publication en \textbf{DocBook}
      \item Mise en place de \textit{workflows} de validation
      \item Gestion des versions sous Subversion
      \item Gestion des tickets sous Trac
      \item Rédaction de la documentation en anglais
      \item Encadrement de trois stagiaires en alternance et de deux traducteurs
    \end{itemize}}

%------------------------------------------------

\cventry
    {2004 – 2006}
    {Dolphian}
    {Solution antispam}
    {}
    {}
    {Guides d'installation, d'utilisation et de référence en anglais.
      \begin{itemize}
      \item Analyse des cahiers des charges et des spécifications et interviews
        des développeurs
    \end{itemize}}

%------------------------------------------------

\subsection{Rédacteur technique et marketing}

\cventry
    {2000 – 2004}
    {Ring}
    {Logiciels de téléphonie}
    {}
    {}
    {
      \begin{itemize}
      \item Guides de référence et d'utilisation en anglais.
      \item Réalisation d'une animation en Flash (scénario, dialogue,
        \textit{storyboard}, suivi de projet)
    \end{itemize}}

%------------------------------------------------

\cventry
    {1996 – 1999}
    {Loan System-Sysload}
    {Logiciels système et financiers}
    {}
    {}
    {
      \begin{itemize}
      \item Guides d'installation et d'utilisation et aide en ligne en anglais
        et en français
      \item Création et rédaction d'un magazine d'entreprise en 3 langues
        (anglais, français, allemand)
      \item Conception, rédaction et intégration HTML d'un site
        \textit{corporate}
      \item Rédaction de brochures marketing et de \textit{flyers}
      \end{itemize}
    }

%------------------------------------------------

\subsection{Chef de projet en traduction}

\cventry
    {1993 – 1996}
    {WordPerfect}
    {Suite bureautique}
    {}
    {}
    {
      \begin{itemize}
      \item Localisation d'interfaces de logiciels
      \item Traduction d'aides en ligne, de manuels, de brochures et de flyers
      \end{itemize}
    }

%-------------------------------------------------------------------------------

\cventry
    {1991 – 1993}
    {Gecap}
    {}
    {}
    {}
    {
      \begin{itemize}
      \item Traduction d'aides en ligne et de manuels
      \end{itemize}
    }

%-------------------------------------------------------------------------------
%	AWARDS SECTION
%-------------------------------------------------------------------------------

\section{Récompenses}

\cvitem{1999}{Obtention de 3 prix de la \textit{Society for Technical
    Communication}}

%-------------------------------------------------------------------------------
%	COMPUTER SKILLS SECTION
%-------------------------------------------------------------------------------

\section{Informatique}

\cvitem
    {Formats}
    {DITA XML, ReStructuredText, DocBook}

\cvitem
    {Frameworks}
    {DITA Open-Toolkit, Python Sphinx, FrameMaker, MS Office}

\cvitem
    {Gestion de bugs}
    {Bugzilla, Trac}

\cvitem {Languages} {Scripts Shell, XSLT, CSS, \LaTeX (LaTeX), HTML, notions de
  C}

\cvitem
    {Utilitaires}
    {gmake, awk, sed, expressions rationnelles (expressions
      régulières)}

%-------------------------------------------------------------------------------
%	LANGUAGES SECTION
%-------------------------------------------------------------------------------

\section{Langues}

\cvitem
    {Anglais}
    {Bilingue}

\cvitem
    {Espagnol}
    {Bonne maîtrise}

%-------------------------------------------------------------------------------
%	EDUCATION SECTION
%-------------------------------------------------------------------------------

\section{Formation}

\cventry{1990}{ESIT}{École supérieure d'interprètes et de traducteurs}{}{}{}
\cventry{1987}{Licence d'anglais}{ Paris IV Sorbonne}{}{}{}

%-------------------------------------------------------------------------------
%	INTERESTS SECTION
%-------------------------------------------------------------------------------

\section{Activités}

\cvitem
    {Aïkido}
    {4\textsuperscript{e} dan}

\cvitem
    {Photographie}
    {Photos illustratives pour le site \textit{www.communication-technique.com}}

\cvitem
    {Web}
    {Création d'un site de méthodologie de rédaction technique (sources sous
      \textbf{GitHub}).}

\end{document}
